
uRPG est un systeme de regles minimaliste concue pour etre maitrisé rapidement en proposant un certain comprimis entre la simplicité et le réalisme.

\section{Le jeu de role, rappels et teminologie}

Un jeu de rôle (\textbf{JdR}) est...


Le Maitre du jeu (\textbf{MJ}) est le scénariste et méteur en scene de la partie. Il prépare le scénario de la partie dans lequel évoluent les différents personnages.
Chaque autre joueur incarne son Personnage Joueur (\textbf{PJ}) le \textbf{MJ} incarnant tout les autres Personnages Non Joueurs i(\textbf{PNJ}).
Ces differents personnages seront nemés a interragir et a entreprendre des actions.
Le systeme de regles permet de determiner le success de ces actions, la quasi totalité des systemes exiqstant reposent sur des lancés de dés.

uRPG utilise uniquement des dés à 6 face.



Un jeu est caractérisé par:

-> le systeme de regles utilisé: du plus simple sans aucun dés aux plus complexes, il en existe pour tout les gouts. uRPG a vocation a etre facilement maitrisé et proposer une résolution d'actions rapide pour ne pas casser le rythme de la partie.  

-> l'univers dans lequel il se déroule: uRPG est utilisable dans n'importe quel univers raisonablement réaliste: il ne prévoit pas, en l'etat, de pouvoirs magiques ou de capacités surhumaines.


\section{Déroulement d'une partie}

\subsection{En amont, pour le \textbf{MJ}}

Le \textbf{MJ} prépare le scénario de la partie, 

\subsection{En amont, pour les joueurs}


Une fois familiarisés avec l'univers dans lequel se déroulera le jeu, les joueurs vont devoir créer leur \textbf{PJ}: lui inventer une histoire et une personalité; puis completer le fiche de personnage.



\subsection{La partie}


En général une partie débute par une mise en situation, dans laquelle le \textbf{MJ} explique aux Joueurs ce qu'ilms doivent faire, la quete ou l'enquete à réaliser. Puis les joueurs decident librement des actions de leurs personnages.
Les joueurs étant totalement libres de leurs actions, c'est au \textbf{MJ} de s'assurer qu'ils seront suffisament motivés pour suivre son scénario. 

\section{Materiel necessaire}

uRPG utilise un grand nombre de dés a 6 faces, il faut en prévoir une vingtaine, si possible, par joeuur. Il existe des petits dés de moins d'un centimetre de coté qui faciliteront les gros lancés. 

Le \textbf{MJ} peut pouvoir montrer des plans de lieux aux joueurs, un dessin sur une feuille peut suffir mais le choix de support est vaste: tableau effacable, vidéoprojecteur sur la table etc..


 







