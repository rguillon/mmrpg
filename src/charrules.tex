
\section{Les personnages} 

Chaque personnage est représenté par:

- Ses caractéristiques, qui déterminent ses capacités naturelles.

- Ses compétences et ses équipements qui vont l'aider a réaliser ses tâches.

- Ses points de santé physique et mentale sont sa capacité à encaisser les coups et le stress, ainsi que les dégâts pour chacunes.


\section{Les caractéritiques}

Il y a quatre caractéristiques:

- \textbf{CONS}: La constitution est la force physique du personnage, sa capacité a soulever des charges ou a frapper.

- \textbf{DEXT}: La dextérité représente l'adresse et l'agilité.

- \textbf{INTE}: L’intelligence est la capacité a comprendre, résoudre des problemes et apprendre.

- \textbf{CHAR}: Le charisme est la TODO

Chaque jet d'action va utiliser l'une des quatre caractéristiques. En cas s'ambiguité, le MJ tranche. Le jet d'action consiste a lancer autant de dés que le niveau de la caractéristique utilisée.
	
\example{pour grimper sur le toit d'un bâtiment, un personnage peut utiliser sa force ou sa dextérité, en accord avec le MJ.}

\section{Les compétences}

Il faut dépenser N*(N+1)/2 points pour compétence de niveau N. De nouveaux points de compétences peuvent être acquis en cours de jeu, ils peuvent servir a obtenir de nouvelles compétences ou en améliorer des existantes.

\example{Avec une agilité à 4 et une compétence d'escalade a 3, le personnage dispose de 7 dés pour tenter de grimper sur le toit du bâtiment.}

\section{Les équipements}

Comme les compétences, les équipement procurent des bonus rattachés a une caractéristique. Un seul équipement peut être utilisé par jet. Le coût en points d'équipements est le même que pour les compétences: N*(N-1)/2 pour un effet de niveau N. Un équipement peut éventuellement disposer de plusieurs bonus différents, mais un seul est utilisable par jet, le coût total est le cumul des coûts de chaque bonus. 

Les points d'équipement représentent l'argent, le MJ indique le taux de change entre un point et la monnaie utilisée.

\example{avec une agilité a 4, une compétence escalade à 3 et une corde et baudrier à DEXT+4, le personnage dispose désormais de 11 dés.}

Un équipement peut disposer d'un effet permanent, qui va directement modifier l'une des caractéristiques. Ces bonus permanents peuvent être des malus et coutent 6 fois le prix classique.

\example{une très grosse épée a un bonus de +7 en attaque et -1 permanent en DEXT: lors de son utilisation, le bonus effectif est donc de +6, en revanche elle impose un -1 a tout les jets de dextérité: elle est tellement encombrante qu'elle gène son utilisateur.}

\section{La résolution des actions}
	
Les actions se résolvent par jet de dés à 6 faces de la maniéré suivante: 

* Un 1,2, 3 ou 4 sont des échecs
* Un 5 rapporte 1 succès
* Un 6 rapporte 2 succès
	
Chaque dé a donc une espérance d'un demi succès.
	
Le nombre de dés a lancé est déterminé par: le niveau de la caractéristique, de l’équipement et de la compétence utilisés moins la somme des points de dégâts (physique et mentale).
	
\subsection{Actions simples}
Le personnage effectue son jet, totalise ses succès et indique son score au MJ qui décide si l'action est réussie.
	
\example{avec une agilité a 4, une compétence escalade à 3 et une corde et baudrier à DEXT+4, le personnage dispose désormais de 11 dés. Il effectue son jet et n'obtient que 3 succès, trop peu au goût du MJ qui décide que le personnage chute et prends deux points de blessure. S'il fait une seconde tentative, il n'aura plus que 9 dés.}

\subsection{Action en opposition}
Les deux personnages effectuent leurs jets. Le plus haut score l'importe.

\subsection{Combats}
Il s'agit d'une action en opposition particulière. Un combat est divisé en rounds ou chaque personnage en jeu est l’attaquant et cible une victime. 
L’attaquant et un défenseur font un jet. Si le score de l'attaquant dépasse celui du défenseur, ce dernier prends l’écart des scores en points de dégâts. C’est au tour du joueur suivant d’être l’attaquant.

Sauf embuscade, l’odre des attaquant est determiné par le niveau de dextérité: le plus habile frappe le premier.

Si le nombre de points de dégâts physique devient supérieur ou égal au points de vie, le personnage meurs. Si ses dégâts psychologiques dépassent sa santé mentale, il devient fou et le MJ en prends le contrôle.
