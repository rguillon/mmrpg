\begin{landscape}
\begin{multicols}{4}


\date{} % clear date
\maketitle{}


\si\micro RPG est un systeme de règles minimaliste, symétrique dont les actions se résolvent par jet de (nombreux) dés à 6 faces.
Les regles se veulent relativement rélistes et sont plus adaptées pour representer des personnages ordinaires que des héros.

La premiere feuille, recto-verso, donne les regles et la feuille de personnage, elle est suffisante pour un personnage joueur.
Le seconde donne des indications pour le maitre du jeu ainsi que des feuillles de personnages non-joueurs.

\section{Les personnages} 


\subsection{Les caractéritiques}

Chaque personnage dispose de quatre caractéristiques:


$\rightarrow$ \textbf{CONST}: La constitution est la force physique du personnage, sa capacité a soulever des charges ou a frapper.

$\rightarrow$ \textbf{DEXT}: La dextérité  est l'adresse et l'agilité.

$\rightarrow$ \textbf{INT}: L’intelligence est la capacité a comprendre, résoudre des problemes et apprendre.

$\rightarrow$ \textbf{CHAR}: Le charisme est la capacité a séduire ou intimider.


Un joueur dispose de \textbf{16 points} à répartir dans ces caractéristiques, puis il en déduit ses points:


$\rightarrow$ Points de santé physique:

\textit{CONST+DEXT+6}

Representent le nombre de dégats physique qu'il peut subir.


$\rightarrow$ Points de santé mentale:

\textit{INT+CHAR+6}

Represente la capacité a subir du stress et les situations angoissantes.



$\rightarrow$ Points de compétence:

\textit{INT+DEXT+6}

A dépenser en compétences.


$\rightarrow$ Points d'équipement:	

\textit{CONST+CHAR+6}

Representent l'argent, à dépenser en équipement. 
	

\subsection{Les compétences}

Chaque compétence, representée par un verbe, a un niveau qui rajoute autant de dés au jet lorsqu'elle est utilisée.
	
Il faut dépenser \textit{N*(N+1)/2} points pour compétence de niveau N.


\subsection{Les équipements}

Les équipements ont également des effets qui rajoutent des dés quand ils sont utilisés. 
Comme les compétences, un effet de niveau N coute \textit{N*(N+1)/2} points d'équipements.
Un équipement utilisable de plusieurs façons peut disposer de plusieurs effets, mais un seul est utilisable par jet. 

Un équipement peut avoir des effets pérmanants qui vont directement affecter l'une des caractéristiques tant que le personnage en est équipé.
Ces effets permanents ont un cout 6 fois plus élévé et ils peuvent etre négatifs, pour representer une gene ou un handicap.


Le coût total d'un équipement est la somme des couts de chaque effet.

Enfin il peut y avoir des modificateurs de couts à la discretion du MJ, si l'équipement n'est utilisable que dans des contitions tres particulieres.


\section{Les jets d'actions}
	
Les actions se résolvent par jet de dés à 6 faces: 

- Un 1,2, 3 ou 4 sont des échecs

- Un 5 rapporte 1 succès

- Un 6 rapporte 2 succès
	
Le nombre de dés a lancer est déterminé par: le niveau de la caractéristique, de l'effet de l’équipement et de la compétence utilisés; moins la somme des points de dégâts physiques et mentaux.

Si les dégats sont trop importants et qu'il ne reste aucun dés a lancer; alors le personnage ne peut pas effectuer l'action, il est trop mal en point.


	
\subsection{Actions simples}

Pour une action simple, le MJ décide en secret du nombre de succès necessaires.
Le personnage effectue son jet et totalise ses succès, le MJ annonce alors si l'action est réusie et ses conséquences. 
	

\subsection{Combats}

Les combats se déroulent par tours, ou chaque personnage en jeu va choisir une cible et l'attaquer.
L'attaquant fait un jet d'attaqué, sa cible fait un jet de défence, si le score de l'attaquant est supérieur, l'attaqué subit l'écart des deux jets en points de dégats physique.
Lorsque chaque personnage a fait son jet d'attaque, un nouveau tour commence.
Le combat continue jusqu'a ce que tout les personnages d'un camp meurent, se rendent ou prenent la fuite.


Sauf embuscade, l’odres des attaquant est determiné par le niveau de \textit{DEXT+INT}: le plus habile et plus malin attaque le premier.

\subsection{Mort}

Si le nombre de points de dégâts physique devient supérieur ou égal au points de vie, le personnage meurs.

Si ses dégâts psychologiques dépassent sa santé mentale, il devient fou, si c'est un personnage joueur, le MJ en prends le contrôle.

\vfill\null

\end{multicols}
\end{landscape}
